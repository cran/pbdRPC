
\section[An Advance Application with \pkg{pbdMPI}]{An Advance Application with \pkg{pbdMPI}}
\label{sec:advance_pbdmpi}
\addcontentsline{toc}{section}{\thesection. An Advance Application with \pkg{pbdMPI}}

Examples of the \pkg{pbdMPI}~\citep{Chen2012pbdMPIpackage} are introduced
in this section first under a shell/terminal mode.
Then, the examples will be combined with \pkg{pbdRPC} to show how
\code{srpc()} sends requests from an interactive \proglang{R} session
to a remote server and execute the examples in the shell/terminal model.
Multiple commands can be manually combined in one \code{srpc()} call.
The return values can also be captured by the interactive \proglang{R}
session with an addition argument.

The \pkg{pbdMPI} is a general MPI interface
running in SPMD by default. A typical example is to run the \pkg{pbdMPI}
via \code{mpiexec} and \code{Rscript} from a shell/terminal as
in below with outputs.
\begin{Code}[title=\code{pbdMPI::allreduce(1)} in 4 cores]
$ mpiexec -np 4 Rscript -e 'library(pbdMPI,quietly=T);allreduce(1);finalize()'
[1] 4
[1] 4
[1] 4
[1] 4
\end{Code}
The outputs are printed from 4 cores. Each core has a \code{allreduce(1)}
call synchronically to reduce three 1's from other peer cores. The default
operation for \code{allreduce()} is a summation \code{sum()}, so the total
is 4. Note that the 4 cores may not be in a single machine as long as MPI
setup correctly.

The example below will give the total 8 in each core because
the value to be reduced is a 2 from each core.
\begin{Code}[title=\code{pbdMPI::allreduce(2)} in 4 cores]
$ mpiexec -np 4 Rscript -e 'library(pbdMPI,quietly=T);allreduce(2);finalize()'
[1] 8
[1] 8
[1] 8
[1] 8
\end{Code}
Similarly, the total will be 12 in each core when 3's are reduced from 4 cours.

With the examples above, one may want to execute them from a local
machine/laptop. i.e.
\pkg{pbdMPI} is run on remote servers while \pkg{pbdRPC} is run in local.
Note that two systems between server and local machines are generally
different.

Assume environment setups are similar as Sections~\ref{sec:application_rr} and
\ref{sec:application_cs}.
The \pkg{pbdRPC} commands within an interactive \proglang{R} session
are shown below.
\begin{Code}[title=\pkg{pbdRPC} in local and \pkg{pbdMPI} in remote]
> library(pbdRPC)
>
> ### Alter login information as needed
> rpcopt_set(user = "snoweye", hostname = "192.168.56.101")
> .pbd_env$RPC.CT$use.shell.exec <- FALSE
>
> ### Set the RPC commands
> preload <- "source ~/work-my/00_set_devel_R; "
> cmd.mpi <- "mpiexec -np 4 Rscript -e "
> cmd.code <- "'library(pbdMPI,quietly=T);allreduce(3);finalize()'"
>
> ### Put the RPC commands together
> cmd <- paste(preload, cmd.mpi, cmd.code, sep = "")
> cmd  ### Similar to the shell example above
>
> ### Send the command to remote server, snoweye@192.168.56.101
> srpc(cmd = cmd)
[1][1] 12
[1] 12
[1] 12
12
>
> ### Turn on verbose
> .pbd_env$RPC.CT$verbose <- TRUE
>
> ### Capture the return values
> ret <- srpc(cmd = cmd, intern = TRUE)
C:/Uners/snoweye/Documents/R/win-library/3.4/pbdRPC/libs/x64/plink.exe -P 22 -i ./id_rsa.ppk  snoweye@192.168.56.101 "source ~/work-my/00_set_devel_R; mpiexec -np 4 Rscript -e 'library(pbdMPI,quietly=T);allreduce(3);finalize()'"
> str(ret)
 chr [1:4] "[1] 12" "[1] 12" "[1] 12" "[1] 12"
> ret
[1] "[1] 12" "[1] 12" "[1] 12" "[1] 12"
\end{Code}
In order to capture the return values (four character strings) from the
remote server, the argument \code{intern = TRUE} set to the \code{srpc()}
is required as shown in the example.

When \code{verbose = TRUE}, the \code{srpc()} shows the command being passed
to the shell or terminal. In this case, one may test the message shown above
to obtain the same result from the interactive \proglang{R} session as
below.
\begin{Command}[title=From a \code{cmd.exe} command prompt windows]
C:\Users\snoweye> cd Documents
C:\Users\snoweye\Documents> 
C:/Users/snoweye/Documents/R/win-library/3.4/pbdRPC/libs/x64/plink.exe -P 22 -i ./id_rsa.ppk  snoweye@192.168.56.101 "source ~/work-my/00_set_devel_R; mpiexec -np 4 Rscript -e 'library(pbdMPI,quietly=T);allreduce(3);finalize()'"
[1] 12
[1] 12
[1][1] 12
 12
\end{Command}
Note that the slash symbols $/$ may be replaced by anti-slash or back slash
symbols $\setminus$ in MS Windows when Rtools is not loaded correctly.
Double quote $"$ may be needed as well when the command path to the
\code{plink.exe} contains any spaces or any non-ascii characters.

